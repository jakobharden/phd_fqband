% !BIB TS-program = biblatex
% !TeX spellcheck = en_US
%
%#######################################################################################################################
% LICENSE
%
% "tmpl_presentation.tex" (C) 2024 by Jakob Harden (Graz University of Technology) is licensed under a Creative Commons Attribution 4.0 International license.
%
% License deed: https://creativecommons.org/licenses/by/4.0/
% Author email: jakob.harden@tugraz.at, jakob.harden@student.tugraz.at, office@jakobharden.at
% Author website: https://jakobharden.at/wordpress/
% Author ORCID: https://orcid.org/0000-0002-5752-1785
%
% This file is part of the PhD thesis of Jakob Harden.
%#######################################################################################################################
%
% Beamer documentation: https://www.beamer.plus/Structuring-Presentation-The-Local-Structure.html
%
% preamble
\documentclass[11pt,aspectratio=169]{beamer}
\usepackage[utf8]{inputenc}
\usepackage[LGR,T1]{fontenc}
\usepackage[ngerman,english]{babel}
\usepackage{hyphenat}
\usepackage{lmodern}
\usepackage{blindtext}
\usepackage{multicol}
\usepackage{graphicx}
\usepackage{tikz}
\usetikzlibrary{calc,fpu}
\usepackage{pgfplots}
\pgfplotsset{compat=1.17}
\usepgflibrary{fpu}
\usepackage{circuitikz}
\usepackage{amsmath}
\usepackage{algorithm}
\usepackage{algpseudocode}
\usepackage{hyperref}
\usepackage[backend=biber,style=numeric]{biblatex}
\addbibresource{biblio.bib}
%
% text blocks
\def\PresTitle{Automated frequency band estimation for sinusoidal low-pass signals}
\def\PresSubTitle{A numerical study on synthetic and natural signals}
\def\PresDate{10/12/2024}
\def\PresFootInfo{My PhD thesis, research in progress ...}
\def\PresAuthorFirstname{Jakob}
\def\PresAuthorLastname{Harden}
\def\PresAuthor{\PresAuthorFirstname{} \PresAuthorLastname{}}
\def\PresAuthorAffiliation{Graz University of Technology}
\def\PresAuthorAffiliationLocation{\PresAuthorAffiliation{}, Graz, Austria}
\def\PresAuhtorWebsite{jakobharden.at}
\def\PresAuhtorWebsiteURL{https://jakobharden.at/wordpress/}
\def\PresAuhtorEmailFirst{jakob.harden@tugraz.at}
\def\PresAuhtorEmailSecond{jakob.harden@student.tugraz.at}
\def\PresAuhtorEmailThird{office@jakobharden.at}
\def\PresAuthorOrcid{0000-0002-5752-1785}
\def\PresAuthorOrcidURL{https://orcid.org/0000-0002-5752-1785}
\def\PresAuthorLinkedin{jakobharden}
\def\PresAuthorLinkedinURL{https://www.linkedin.com/in/jakobharden/}
\def\PresCopyrightType{ccby} % one of: copyright, ccby, ccysa
%
% Beamer theme adaptations
%   type:        Presentation
%   series:      Research in progress (RIP)
%   description: This theme is designed to present preliminary research results.
% !BIB TS-program = biblatex
% !TeX spellcheck = en_US
%
%#######################################################################################################################
% LICENSE
%
% "adaptthemePresRIP.tex" (C) 2024 by Jakob Harden (Graz University of Technology) is licensed under a Creative Commons Attribution 4.0 International license.
%
% License deed: https://creativecommons.org/licenses/by/4.0/
% Author email: jakob.harden@tugraz.at
% Author website: https://jakobharden.at/wordpress/
% Author ORCID: https://orcid.org/0000-0002-5752-1785
%
% This file is part of the PhD thesis of Jakob Harden.
%#######################################################################################################################
%
% Beamer theme adaptations
%   type:        Presentation
%   series:      Research in progress (RIP)
%   description: This theme is designed to present preliminary research results.
%
% Beamer documentation: https://www.beamer.plus/Structuring-Presentation-The-Local-Structure.html
%
%-----------------------------------------------------------------------------------------------------------------------
% color definitions
\definecolor{RIPbgcol}{RGB}{255, 233, 148} % LibreOffice, Light Gold 3
\definecolor{RIPsepcol}{RGB}{255, 191, 0} % LibreOffice, Gold
\definecolor{RIPtitlecol}{RGB}{120, 75, 4} % LibreOffice, Dark Gold 3
%
% geometry definition
\newlength{\RIPheadheight}
\setlength{\RIPheadheight}{14mm}
\newlength{\RIPfootheight}
\setlength{\RIPfootheight}{9mm}
%
%-----------------------------------------------------------------------------------------------------------------------
% commands
% two-column mode, left column
\newenvironment{RIPcolleft}{%
		\begin{column}{.65\textwidth}%
	}{%
		\end{column}%
	}
%
% two-column mode, right column
\newenvironment{RIPcolright}{%
		\hspace{.05\textwidth}%
		\begin{column}{.3\textwidth}%
	}{%
		\end{column}%
	}
%
% copyright information text block
\newcommand{\RIPcopyrightinfo}[1]{%
	Copyright \textcopyright{} \the\year{} \PresAuthor{} (\PresAuthorAffiliationLocation)\\
	%This document is licensed under a Creative Commons Attribution 4.0 International license.
	\ifstrequal{#1}{copyright}{All rights reserved.}{}
	\ifstrequal{#1}{ccby}{%
		This document is licensed under a Creative Commons Attribution 4.0 International license.\\
		See also: \href{https://creativecommons.org/licenses/by/4.0/deed}{CC BY 4.0, license deed}
	}{}%
	\ifstrequal{#1}{ccbysa}{%
		This document is licensed under a Creative Commons Attribution-Share Alike 4.0 International license.\\
		See also: \href{https://creativecommons.org/licenses/by-sa/4.0/deed}{CC BY-SA 4.0, license deed}
	}{}
	\\
	\vspace{1em}
	The above license applies to the entire content of this document. Deviations from this license are explicitly marked.
}
%
% author information text block
\newcommand{\RIPauthorinfo}[1]{%
	\renewcommand{\arraystretch}{1.25}
	\begin{tabular}{l l}
		First name & \PresAuthorFirstname{} \\
		Last name & \PresAuthorLastname{} \\
		Affiliation & \PresAuthorAffiliationLocation{} \\
		Website & \href{\PresAuhtorWebsiteURL{}}{\PresAuhtorWebsite{}} \\
		Email & \PresAuhtorEmailFirst{}, \PresAuhtorEmailSecond{}, \PresAuhtorEmailThird{} \\
		ORCID & \href{\PresAuthorOrcidURL}{\PresAuthorOrcid{}} \\
		LinkedIn & \href{\PresAuthorLinkedinURL}{\PresAuthorLinkedin{}}
	\end{tabular}
}
%
%-----------------------------------------------------------------------------------------------------------------------
% define and adapt theme
\usetheme{default} % presentation theme
\useoutertheme{sidebar} % outer theme
\useinnertheme{default} % inner theme
%
% size settings
\setbeamersize{%
	text margin left=5mm,
	text margin right=5mm,
	sidebar width left=0mm,
	sidebar width right=0mm}
%
% headline settings
\setbeamertemplate{headline}{%
	\begin{minipage}[t]{\textwidth}
		\begin{tikzpicture}
			\fill[RIPbgcol] (0,0) -- ++(16, 0) -- ++(0,-\RIPheadheight) -- ++(-16,0) -- cycle;
			\draw[RIPsepcol] (0,-\RIPheadheight) -- ++(16,0)
				node[pos=0.988,left,yshift=.6\RIPheadheight,RIPtitlecol] {\Large\insertpagenumber};
		\end{tikzpicture}
	\end{minipage}
	%\insertpagenumber
}
%
% footline settings
\setbeamertemplate{footline}{%
	\begin{minipage}[t]{\textwidth}
		\begin{tikzpicture}
			\fill[RIPbgcol] (0,0) -- ++(16, 0) -- ++(0,-\RIPfootheight) -- ++(-16,0) -- cycle;
			\draw[RIPsepcol] (0,0) -- ++(16,0)
				node[pos=0.012,black,right,yshift=-4mm] {\small\PresFootInfo{}}
				node[pos=0.988,black,left,yshift=-5mm]{
					\parbox{35mm}{%
						\raggedleft
						\small\hfill\PresAuthor{}\newline
						\tiny\PresAuthorAffiliation{}
					}
				};
		\end{tikzpicture}
	\end{minipage}
}
%
% left sidebar settings
\setbeamertemplate{sidebar canvas left}{}
\setbeamertemplate{sidebar left}{}
%
% nagigation symbol settings
\setbeamertemplate{navigation symbols}{}
%
% abstract settings
\setbeamertemplate{abstract title}{\normalsize}
\setbeamertemplate{abstract begin}{\small}
\setbeamertemplate{abstract end}{\normalsize}
%
% color settings
\setbeamercolor{titlelike}{fg=RIPtitlecol}
\setbeamercolor{bibliography entry author}{fg=RIPtitlecol}
\setbeamercolor{bibliography entry note}{fg=RIPtitlecol}
\setbeamercolor{bibliography item}{fg=black}
\setbeamercolor{caption}{fg=black}
\setbeamercolor{caption name}{fg=RIPtitlecol}
%
% itemization settings
\setbeamertemplate{itemize item}{\color{RIPtitlecol}$\blacktriangleright$}
\setbeamertemplate{itemize subitem}{\color{RIPtitlecol}$\blacksquare$}
%
% enumeration settings
\setbeamertemplate{enumerate item}{\color{RIPtitlecol}\bfseries\insertenumlabel}
%
% bibliography settings
\setbeamertemplate{bibliography item}{\insertbiblabel}

%
% Graphics file path
\graphicspath{ {../../octave/results/test_fqband} }
%
%#######################################################################################################################
\begin{document}
	% set title page items
	\author{\PresAuthor{} (\PresAuthorAffiliation{})}
	\title{\PresTitle{}}
	\subtitle{\PresSubTitle{}}
	%\logo{}
	%\institute{}
	\date{\PresDate{}}
	%\subject{}
	%\setbeamercovered{transparent}
	%\setbeamertemplate{navigation symbols}{}
	%
	%-------------------------------------------------------------------------------------------------------------------
	\begin{frame}[plain]
		\maketitle
	\end{frame}
	%
	%-------------------------------------------------------------------------------------------------------------------
%	Die klassische Analyse von Signalen beginnt üblicherweise damit, möglichst viele Informationen über die Eigenschaften des Signals zu sammeln. Diese Informationen sind die Basis für die sinnvolle Auswahl von Analysemethoden und deren Parametrierung.
%	Bei sinusförmige Signalen ist das Frequenzband des Signals eine dieser herausragenden Signaleigenschaften. Die Abschätzung des Frequenzbands kann insbesondere bei Signalen mit kleiner Signal-to-noise ratio eine Herausforderung darstellen.
%	In dieser Arbeit wird eine vollständig automtisierte Methode vorgestellt mit der das Frequenzband von low-pass Signalen einfach, schnell und stabil abgeschätzt werden kann. Die Abschätzung basiert auf der diskreten Fourier-Transformation und einer Schwellenwertdetektion. Aufgrund der vollständigen Automatisierung wird hier besonderes Augenmerk auf die Stabilität der Methode gelegt.
%	Die Leistungsfähigkeit der Methode wird hier anhand von synthetischen und natürlichen Signalen überprüft. Die vorläufigen Ergebnisse zeigen, dass die automatisierte Abschätzung und die visuelle Erkennung vergleichbare Ergebnisse liefern.
	\section*{Abstract}
	\begin{frame}
		\frametitle{Abstract}
		\begin{abstract}
			Classic signal analysis usually begins with collecting information about the properties of the signal to the greatest extent possible. This information is the basis for the sensible selection of analysis methods and their parameterization.
			
			For sinusoidal signals, the frequency band of the signal is one of these fundamental signal properties. Estimating the frequency band can be a challenge, especially for signals with a small signal-to-noise ratio.
			
			This work proposes a fully automated method for easily, quickly, and stably estimating the frequency band of low-pass signals. The estimation is based on the discrete Fourier transform and threshold detection. Due to the complete automation, special attention is paid to the method's stability.
			
			The performance of the method is tested here using synthetic and natural signals. The preliminary results show that the automated estimation and visual detection deliver comparable results for the upper frequency band limit.
		\end{abstract}
	\end{frame}
	%
	%-------------------------------------------------------------------------------------------------------------------
	\section{Introduction}
	\begin{frame}
		\frametitle{Introduction}
		% motivate the topic and/or problem
		% reason why this work is useful, use cases
		% briefly describe how to solve the problem
		\begin{itemize}
			\item \textcolor{RIPtitlecol}{WHAT}
			\begin{itemize}
				\item Estimate upper frequency band limit for damped sinusoidal low-pass signals.
			\end{itemize}
			\item \textcolor{RIPtitlecol}{WHY}
			\begin{itemize}
				\item Gain fundamental signal information for the subsequent signal analysis.
			\end{itemize}
			\item \textcolor{RIPtitlecol}{HOW}
			\begin{itemize}
				\item Estimate power spectral density of the signal (PSD).
				\item Threshold detection using the cumulative sum of the PSD.
			\end{itemize}
			\item \textcolor{RIPtitlecol}{USAGE}
			\begin{itemize}
				\item Synthetic and natural low-pass signals (page \pageref{res:synthetic} ff).
				\item Signals from ultrasonic pulse transmission tests (page \pageref{res:natural} ff).
			\end{itemize}
			\item \textcolor{RIPtitlecol}{HIGHLIGHT}
			\begin{itemize}
				\item \textbf{Automated estimation and the visual inspection deliver comparable results.}
			\end{itemize}
		\end{itemize}
	\end{frame}
	%
	%-------------------------------------------------------------------------------------------------------------------
	\section{Materials \& Methods}
	\begin{frame}
		\frametitle{Materials \& Methods}
		% a brief summary of ...
		%   o  materials
		%   o  methods
		%   o  most important assumptions/parameters
		\begin{itemize}
			\item \parbox{20mm}{\textcolor{RIPtitlecol}{Materials}}
				\begin{itemize}
					\item Synthetic signals: damped sinusoidal low-pass signals in noise (for testing purposes)
					\item Natural signals from ultrasonic pulse transmission tests (cement paste, air, water)
				\end{itemize}
			\item \parbox{20mm}{\textcolor{RIPtitlecol}{Methods}}
				\begin{itemize}
					\item Power spectral density (PSD) estimation
					\item Cumulative sum
					\item Threshold detection
				\end{itemize}
			\item \parbox{20mm}{\textcolor{RIPtitlecol}{Assumptions}}
			\begin{itemize}
				\item signals are (mostly) mean-free, (damped) sinusoidal low-pass signals
				\item uncorrelated noise, i.i.d. noise, Gaussian white noise
			\end{itemize}
		\end{itemize}
	\end{frame}
	%
	%-------------------------------------------------------------------------------------------------------------------
	\begin{frame}
		\frametitle{Materials - synthetic test signals}
		\begin{itemize}
			\item \textcolor{RIPtitlecol}{Synthetic test signals - damped sinusoidal signals in noise}
			\begin{itemize}
				\item Signal model: $x = s + \nu$
				\item $s[n] = A \cdot \left( \sum\limits_{F=1}^{5} sin(2\,\pi\,F\,\frac{n}{N-1}) \right) \cdot e^{-DF\,\frac{n}{N-1}}$
				\item $n = 0,\ldots,N-1, \quad N = F_s \cdot N_{cy}$
				\item Noise $\nu$, i.i.d. noise, Gaussian white noise
				\item Signal frequencies $F = [1, 2, 3, 4, 5]$ [Hz], Sampling frequency $F_s = 100$ [Hz]
				\item Amplitude $A = 1$ V, number of cycles $N_{cy} = 3$, exponential decay factor $DF$
				\item Noise floor before and behind the middle signal section, $L_{pre} = L_{post} = F_s$
			\end{itemize}
			\item \textcolor{RIPtitlecol}{Analysis settings:}
			\begin{itemize}
				\item Frequency threshold factor $t_F = 0.75$ (noise measurement range: 25 to 100 Hz)
				\item Power threshold factor $t_P = 0.93$
				\item No FFT zero-padding
			\end{itemize}
		\end{itemize}	
	\end{frame}
	%
	%-------------------------------------------------------------------------------------------------------------------
	\begin{frame}
		\frametitle{Materials - natural signals}
		\begin{itemize}
			\item \textcolor{RIPtitlecol}{Signal series from ultrasonic pulse transmission tests}
			\begin{itemize}
				\item Specimen: cement paste, water-cement ratio 0.4 (ordinary Portland cement and water)
				\item Measurement distance $D = 50$ mm
				\item Piezo-electric sensor's natural frequency $F_{nat} = 500$ kHz
				\item Pulse voltage 800 [V], Pulse width 2.5 $\mu$Sec
				\item Sampling frequency $F_s = 10$ MHz
				\item Vertical resolution 16 bit
				\item Test duration 24 hr (cement paste tests)
				\item Recording interval 5 Min (cement paste tests)
				\item Two-channel test (P-wave, S-wave)
			\end{itemize}
			\item \textcolor{RIPtitlecol}{Analysis settings:}
			\begin{itemize}
				\item Frequency threshold factor $t_F = 0.9$ (noise measurement range: 1 to 10 MHz)
				\item Power threshold factor $t_P = 0.93$
				\item FFT zero-padding, 3 times signal length
			\end{itemize}
		\end{itemize}	
	\end{frame}
	%
	%-------------------------------------------------------------------------------------------------------------------
	\begin{frame}
		\frametitle{Method - signal processing schema}
		\begin{figure}
			\begin{circuitikz}[scale=3, european]
				\def\nodedist{0.8}
				\draw
				(0, 1)
					node[mixer,scale=0.5] (M1) {}
				++(-0.4, 0)
					node[left] {$x_{LP}$}
					to[short]
				(M1.w)
					node[inputarrow] {}
				(M1.s)
					node[inputarrow,rotate=90] {}
					to[short]
					++(0,-0.25)
					node[below] {$window$}
				(M1.e)
					to[twoport,>,t=FFT]
					node[above,pos=1] {$X_{LP}$}
				++(\nodedist,0)
					to[twoport,>,t=PSD,a=unilateral,n=PSD]
					node[above,pos=1] {$S_{xx}$}
				++(\nodedist,0)
					to[twoport,>,t=CS,l=\parbox{2cm}{\centering cumulative \\ sum\\ \vspace*{1em}}]
					node[above,pos=1] {$C_S$}
				++(\nodedist,0)
					to[twoport,>,t=TD,n=TD,l=\parbox{2cm}{\centering threshold \\ detection\\ \vspace*{1em}}]
				++(\nodedist,0)
					node[inputarrow] {}
					node[right] {$\mathbf{F_{lim}}$}
				(PSD.e)
					-|
				++(0.2,-0.8)
					to[twoport,>,t=PE,n=PE,l=\parbox{2cm}{\centering power \\ estimates\\ \vspace*{1em}}]
				++(1.2,0)
					-|
					node[left,pos=0.75] {$P_s$}
				(TD.s)
					node[inputarrow,rotate=90] {}
				(TD.se)
					node[inputarrow,rotate=90] {}
					to[short]
				++(0,-0.25)
					node[below] {$t_P$}
				(PE.sw)
					node[inputarrow] {}
					to[short]
				++(-0.425,0)
					node[left] {$F_s, \, t_F$}
				(PE.se)
					--
				++(0.425,0)
					node[inputarrow] {}
					node[right] {$(P_x, \, P_{\nu})$};
			\end{circuitikz}
		\end{figure}
	\small $x_{LP}$~\ldots low-pass signal, $F_s$~\ldots Sampling frequency, $t_F$~\ldots frequency threshold factor, $t_P$~\ldots power threshold factor, $P_x$~\ldots power of signal in noise, $P_{\nu}$~\ldots power of noise, $P_s$~\ldots signal power, $F_{lim}$~\ldots estimate upper frequency band limit. See also notes on next page.
	\end{frame}
	%
	%-------------------------------------------------------------------------------------------------------------------
	\begin{frame}
		\frametitle{Method - Signal processing notes}
		\begin{itemize}
			\item \textcolor{RIPtitlecol}{Frequency threshold factor $t_F$}
			\begin{itemize}
				\item multiplier of the Nyquist frequency $F_{ny} = F_s / 2$
				\item used to determine the frequency range to estimate the noise power $P_{\nu}$
				\item can be obtained from initial visual inspection and/or measurement device characteristics
			\end{itemize}
			\item \textcolor{RIPtitlecol}{Power threshold factor $t_P$}
			\begin{itemize}
				\item multiplier of the estimate signal power $P_s = P_x - P_{\nu}$
				\item avoid detection over-run for signals with a high signal-to-noise ratio ($P_{\nu} \approx 0$)
			\end{itemize}
			\item \textcolor{RIPtitlecol}{PSD scaling:} The PSD magnitudes are scaled such that the sum of the magnitudes is equal to the total signal power $P_x$.
			\item \textcolor{RIPtitlecol}{Cumulative sum (CS) characteristics for low-pass signals}
			\begin{itemize}
				\item Frequency 0 to $F_{lim}$: steep curve until approx. signal energy $P_s$ is reached
				\item Frequency $F_{lim}$ to $F_{ny}$: mostly linear increasing curve (slope depends on $P_{\nu}$)
			\end{itemize}
			\item \textcolor{RIPtitlecol}{Algorithm:} see appendix, page \ref{app:algorithm}
		\end{itemize}	
	\end{frame}
	%
	%-------------------------------------------------------------------------------------------------------------------
	\section{Results}
	\begin{frame}
		\frametitle{Results - synthetic test signals}\label{res:synthetic}
		% place graphs and images in the left column
		% comment on results in the right column
		\begin{columns}[t]
			\begin{RIPcolleft}
				\begin{figure}
					\includegraphics[height=55mm,trim= 0mm 0mm 0mm 20mm] {sig_DF_0_SNR_0.png}
				\end{figure}
			\end{RIPcolleft}
			\begin{RIPcolright}
				\textbf{Test signal 1, model:} \\
				\begin{itemize}
					\item no exponential decay, $DF = 0$
					\item low signal-to-noise ratio, $SNR = 0$ dB
				\end{itemize}
			\end{RIPcolright}
		\end{columns}
	\end{frame}
	%
	%-------------------------------------------------------------------------------------------------------------------
	\begin{frame}
		\frametitle{Results - synthetic test signals}
		% place graphs and images in the left column
		% comment on results in the right column
		\begin{columns}[t]
			\begin{RIPcolleft}
				\begin{figure}
					\includegraphics[height=55mm,trim= 0mm 0mm 0mm 20mm] {syn_DF_0_SNR_0.png}
				\end{figure}
			\end{RIPcolleft}
			\begin{RIPcolright}
				\textbf{Test signal 1, results:} \\
				\begin{itemize}
					\item CS: sharp transition at $F_{lim}$
					\item $F_{lim}$ is slightly under-estimated
				\end{itemize}
			\end{RIPcolright}
		\end{columns}
	\end{frame}
	%
	%-------------------------------------------------------------------------------------------------------------------
	\begin{frame}
		\frametitle{Results - synthetic test signals}
		% place graphs and images in the left column
		% comment on results in the right column
		\begin{columns}[t]
			\begin{RIPcolleft}
				\begin{figure}
					\includegraphics[height=55mm,trim= 0mm 0mm 0mm 20mm] {sig_DF_1_SNR_0.png}
				\end{figure}
			\end{RIPcolleft}
			\begin{RIPcolright}
				\textbf{Test signal 2, model:} \\
				\begin{itemize}
					\item low exponential decay, $DF = 1$
					\item low signal-to-noise ratio, $SNR = 0$ dB
				\end{itemize}
			\end{RIPcolright}
		\end{columns}
	\end{frame}
	%
	%-------------------------------------------------------------------------------------------------------------------
	\begin{frame}
		\frametitle{Results - synthetic test signals}
		% place graphs and images in the left column
		% comment on results in the right column
		\begin{columns}[t]
			\begin{RIPcolleft}
				\begin{figure}
					\includegraphics[height=55mm,trim= 0mm 0mm 0mm 20mm] {syn_DF_1_SNR_0.png}
				\end{figure}
			\end{RIPcolleft}
			\begin{RIPcolright}
				\textbf{Test signal 2, results:} \\
				\begin{itemize}
					\item CS: sharp transition at $F_{lim}$
					\item $F_{lim}$ is slightly under-estimated
				\end{itemize}
			\end{RIPcolright}
		\end{columns}
	\end{frame}
	%
	%-------------------------------------------------------------------------------------------------------------------
	\begin{frame}
		\frametitle{Results - synthetic test signals}
		% place graphs and images in the left column
		% comment on results in the right column
		\begin{columns}[t]
			\begin{RIPcolleft}
				\begin{figure}
					\includegraphics[height=55mm,trim= 0mm 0mm 0mm 20mm] {sig_DF_2_SNR_0.png}
				\end{figure}
			\end{RIPcolleft}
			\begin{RIPcolright}
				\textbf{Test signal 3, model:} \\
				\begin{itemize}
					\item moderate exponential decay, $DF = 2$
					\item low signal-to-noise ratio, $SNR = 0$ dB
				\end{itemize}
			\end{RIPcolright}
		\end{columns}
	\end{frame}
	%
	%-------------------------------------------------------------------------------------------------------------------
	\begin{frame}
		\frametitle{Results - synthetic test signals}
		% place graphs and images in the left column
		% comment on results in the right column
		\begin{columns}[t]
			\begin{RIPcolleft}
				\begin{figure}
					\includegraphics[height=55mm,trim= 0mm 0mm 0mm 20mm] {syn_DF_2_SNR_0.png}
				\end{figure}
			\end{RIPcolleft}
			\begin{RIPcolright}
				\textbf{Test signal 3, results:} \\
				\begin{itemize}
					\item CS: sharp transition at $F_{lim}$
					\item $F_{lim}$ is slightly under-estimated
				\end{itemize}
			\end{RIPcolright}
		\end{columns}
	\end{frame}
	%
	%-------------------------------------------------------------------------------------------------------------------
	\begin{frame}
		\frametitle{Results - synthetic test signals}
		% place graphs and images in the left column
		% comment on results in the right column
		\begin{columns}[t]
			\begin{RIPcolleft}
				\begin{figure}
					\includegraphics[height=55mm,trim= 0mm 0mm 0mm 20mm] {sig_DF_4_SNR_0.png}
				\end{figure}
			\end{RIPcolleft}
			\begin{RIPcolright}
				\textbf{Test signal 4, model:} \\
				\begin{itemize}
					\item strong exponential decay, $DF = 4$
					\item low signal-to-noise ratio, $SNR = 0$ dB
				\end{itemize}
			\end{RIPcolright}
		\end{columns}
	\end{frame}
	%
	%-------------------------------------------------------------------------------------------------------------------
	\begin{frame}
		\frametitle{Results - synthetic test signals}
		% place graphs and images in the left column
		% comment on results in the right column
		\begin{columns}[t]
			\begin{RIPcolleft}
				\begin{figure}
					\includegraphics[height=55mm,trim= 0mm 0mm 0mm 20mm] {syn_DF_4_SNR_0.png}
				\end{figure}
			\end{RIPcolleft}
			\begin{RIPcolright}
				\textbf{Test signal 4, results:} \\
				\begin{itemize}
					\item CS: smooth transition at $F_{lim}$
					\item $F_{lim}$ is slightly under-estimated
				\end{itemize}
			\end{RIPcolright}
		\end{columns}
	\end{frame}
	%
	%-------------------------------------------------------------------------------------------------------------------
	\begin{frame}
		\frametitle{Results - natural signals}
		% place graphs and images in the left column
		% comment on results in the right column
		\begin{columns}[t]
			\begin{RIPcolleft}
				\begin{figure}
					\includegraphics[height=55mm,trim= 0mm 0mm 0mm 20mm] {ts_DS_ts1_wc040_d50_4.png}
				\end{figure}
			\end{RIPcolleft}
			\begin{RIPcolright}
				\textbf{Data set:} \\
				\begin{itemize}
					\item ts1\_wc040\_d50\_4 \cite{ts1ds}
					\item 288 signals per channel
				\end{itemize}
				\textbf{Observations:} \\
				\begin{itemize}
					\item P-wave: maximum $F_{lim} \approx 100$ \% of $F_{nat}$
					\item S-wave: maximum $F_{lim} \approx 55$ \% of $F_{nat}$
				\end{itemize}
			\end{RIPcolright}
		\end{columns}
	\end{frame}
	%
	%-------------------------------------------------------------------------------------------------------------------
	\begin{frame}
		\frametitle{Results - natural signals}
		% place graphs and images in the left column
		% comment on results in the right column
		\begin{columns}[t]
			\begin{RIPcolleft}
				\begin{figure}
					\includegraphics[height=55mm,trim= 0mm 0mm 0mm 20mm] {nat_DS_ts1_wc040_d50_4_SID_1.png}
				\end{figure}
			\end{RIPcolleft}
			\begin{RIPcolright}
				\textbf{Data set:} \\
				\begin{itemize}
					\item ts1\_wc040\_d50\_4 \cite{ts1ds}
					\item signal \#1 of 288, maturity 0 hr
				\end{itemize}
			\end{RIPcolright}
		\end{columns}
	\end{frame}
	%
	%-------------------------------------------------------------------------------------------------------------------
	\begin{frame}
		\frametitle{Results - natural signals}
		% place graphs and images in the left column
		% comment on results in the right column
		\begin{columns}[t]
			\begin{RIPcolleft}
				\begin{figure}
					\includegraphics[height=55mm,trim= 0mm 0mm 0mm 20mm] {nat_DS_ts1_wc040_d50_4_SID_24.png}
				\end{figure}
			\end{RIPcolleft}
			\begin{RIPcolright}
				\textbf{Data set:} \\
				\begin{itemize}
					\item ts1\_wc040\_d50\_4 \cite{ts1ds}
					\item signal \#24 of 288, maturity 2 hr
				\end{itemize}
			\end{RIPcolright}
		\end{columns}
	\end{frame}
	%
	%-------------------------------------------------------------------------------------------------------------------
	\begin{frame}
		\frametitle{Results - natural signals}
		% place graphs and images in the left column
		% comment on results in the right column
		\begin{columns}[t]
			\begin{RIPcolleft}
				\begin{figure}
					\includegraphics[height=55mm,trim= 0mm 0mm 0mm 20mm] {nat_DS_ts1_wc040_d50_4_SID_48.png}
				\end{figure}
			\end{RIPcolleft}
			\begin{RIPcolright}
				\textbf{Data set:} \\
				\begin{itemize}
					\item ts1\_wc040\_d50\_4 \cite{ts1ds}
					\item signal \#48 of 288, maturity 4 hr
				\end{itemize}
			\end{RIPcolright}
		\end{columns}
	\end{frame}
	%
	%-------------------------------------------------------------------------------------------------------------------
	\begin{frame}
		\frametitle{Results - natural signals}
		% place graphs and images in the left column
		% comment on results in the right column
		\begin{columns}[t]
			\begin{RIPcolleft}
				\begin{figure}
					\includegraphics[height=55mm,trim= 0mm 0mm 0mm 20mm] {nat_DS_ts1_wc040_d50_4_SID_96.png}
				\end{figure}
			\end{RIPcolleft}
			\begin{RIPcolright}
				\textbf{Data set:} \\
				\begin{itemize}
					\item ts1\_wc040\_d50\_4 \cite{ts1ds}
					\item signal \#96 of 288, maturity 8 hr
				\end{itemize}
			\end{RIPcolright}
		\end{columns}
	\end{frame}
	%
	%-------------------------------------------------------------------------------------------------------------------
	\begin{frame}
		\frametitle{Results - natural signals}
		% place graphs and images in the left column
		% comment on results in the right column
		\begin{columns}[t]
			\begin{RIPcolleft}
				\begin{figure}
					\includegraphics[height=55mm,trim= 0mm 0mm 0mm 20mm] {nat_DS_ts1_wc040_d50_4_SID_288.png}
				\end{figure}
			\end{RIPcolleft}
			\begin{RIPcolright}
				\textbf{Data set:} \\
				\begin{itemize}
					\item ts1\_wc040\_d50\_4 \cite{ts1ds}
					\item signal \#288 of 288, maturity 24 hr
				\end{itemize}
				\textbf{Observations:} \\
				\begin{itemize}
					\item P-wave, PSD: additional side lobe at $\approx 500$ kHz
				\end{itemize}
			\end{RIPcolright}
		\end{columns}
	\end{frame}
	%
	%-------------------------------------------------------------------------------------------------------------------
	\begin{frame}
		\frametitle{Results - natural signals}
		% place graphs and images in the left column
		% comment on results in the right column
		\begin{columns}[t]
			\begin{RIPcolleft}
				\begin{figure}
					\includegraphics[height=55mm,trim= 0mm 0mm 0mm 20mm] {ts_DS_ts1_wc040_d50_4.png}
				\end{figure}
			\end{RIPcolleft}
			\begin{RIPcolright}
				\textbf{Data set:} \\
				\begin{itemize}
					\item ts1\_wc040\_d50\_4 \cite{ts1ds}
					\item 288 signals per channel
				\end{itemize}
				\textbf{Observations:} \\
				\begin{itemize}
					\item P-wave: maximum $F_{lim} \approx 100$ \% of $F_{nat}$
					\item S-wave: maximum $F_{lim} \approx 55$ \% of $F_{nat}$
				\end{itemize}
			\end{RIPcolright}
		\end{columns}
	\end{frame}
	%
	%-------------------------------------------------------------------------------------------------------------------
	\begin{frame}
		\frametitle{Results - natural signals}
		\begin{itemize}
			\item \textcolor{RIPtitlecol}{Additional analysis results (see appendix)}
			\begin{itemize}
				\item Cement paste tests, water-cement ratio 0.4, $D = 25$ mm, page \pageref{app:cem25}
				\item Cement paste tests, water-cement ratio 0.4, $D = 70$ mm, page \pageref{app:cem70}
				\item Reference tests, air, $D = 25$ mm, page \pageref{app:air25}
				\item Reference tests, air, $D = 50$ mm, page \pageref{app:air50}
				\item Reference tests, air, $D = 70$ mm, page \pageref{app:air70}
				\item Reference tests, water, $D = 25$ mm, page \pageref{app:water25}
				\item Reference tests, water, $D = 50$ mm, page \pageref{app:water50}
				\item Reference tests, water, $D = 70$ mm, page \pageref{app:water70}
			\end{itemize}
			\vspace*{4em}
		\end{itemize}	
	\end{frame}
	%
	%-------------------------------------------------------------------------------------------------------------------
	\section{Conclusions}
	\begin{frame}
		\frametitle{Conclusions I}
		% briefly summarize all observations
		The results presented before allow, with respect to the chosen analysis parameters, for the following conclusions:
%		In der Kumulationssumme wird der Übergang bei der Grenzfrequenz $F_{lim}$ mit zunehmendem Dämpfungsfaktor runder. Dies hat Auswirkungen auf die Schwellenwertdetektion.
%		Die Grenzfrequenz wird generell geringfügig überschätzt. Erst bei starker exponentieller Dämpfung (DF = 4) wird diese unterschätzt.
%		Die Methode liefert auch bei dominantem Signalrauschen (SNR = 0 dB) zuverlässig Schätzungen für die Grenzfrequenz. Auch dort, wo eine Schwellenwertdetektion im Leistungsspektrum kaum noch möglich ist.
		\begin{itemize}
			\item \parbox{50mm}{\textcolor{RIPtitlecol}{Synthetic signals}}
			\begin{itemize}
				\item In the cumulative sum, the transition at the frequency limit $F_{lim}$ becomes smoother with increasing decay factor. This smoothness is related to the exponential decay and spectral leakage. It also has an impact on the threshold detection.
				\item The frequency limit is generally slightly underestimated.
				\item The method delivers usable frequency limit estimates even when noise is dominant (SNR = 0 dB) and threshold detection in the PSD becomes difficult.
			\end{itemize}
		\end{itemize}
	\end{frame}
		\begin{frame}
		\frametitle{Conclusions II}
		% briefly summarize all observations
		%		Die Methode zeigt auch bei natürlichen Signalen mit großer Bandbreite bezüglich der Frequenzen, Signalrauschen und der Dämpfung gute Stabilität.
		%		Die Grenzfrequenz $F_{lim}$ scheint hier mit der Resonanzfrequenz des Sensors in Zusammenhang zu stehen und ist weitgehend durch diese limitiert.
		%		In den Signalreihen der Zementleim-Tests sind im Leistungsspektrum gelegentlich zusätzliche Seitenbögen zu beobachten, die im Bereich der Resonanzfrequenz und darüber liegen. Dies deutet darauf hin, dass der Sensor in seiner Resonanzfrequenz frei schwingen kann. Die Möglichkeit, dass der Probekörper wenigstens teilweise vom Sensor entkoppelt ist, ist dabei nicht auszuschließen.
		\begin{itemize}
			\item \parbox{50mm}{\textcolor{RIPtitlecol}{Natural signals}}
			\begin{itemize}
				\item The method also shows good stability with natural signals and their broad bandwidth regarding frequencies, noise and decay.
				\item For the vast majority of signals the frequency band limit $F_{lim}$ is limited by the sensor's natural frequency.
				\item In the signal series of the cement paste tests, additional side lobes in the power spectrum can occasionally be observed in the range of the sensor's natural frequency and above. This indicates that the sensor can oscillate freely at its natural frequency. However, the possibility that the test specimen is at least partially decoupled from the sensor surface cannot be ruled out.
			\end{itemize}
		\end{itemize}
		Another observation is that the computation of an entire series of about 600 signals takes about one second. For all results shown here, no computation faults happened.
	\end{frame}
	%
	%-------------------------------------------------------------------------------------------------------------------
	\section{Outlook}
	\begin{frame}
		\frametitle{Outlook}
		% briefly describe further and connected research
		% Nachdem die Methode in ihren Grundzügen beschrieben ist und erste Ergebnisse vorliegen, stellen sich dennoch einige weitere Fragen.
		Now that the method is described in its basic principles and first results are available, further questions still arise.
		\begin{itemize}
			\item How do estimation results depend on the spectral leakage of the discrete Fourier transformation? Spectral leakage arises from exponential decay and non-coherent sampling.
			\item How does the amount of signal data influence estimation accuracy and precision? Power estimates, especially the noise power, depends on the amount of available data.
			\item How do the results change when changing the method's estimation schema towards accuracy? At the moment, the power of noise in the low-pass section is not taken into account.
		\end{itemize}
	\end{frame}
	%
	%-------------------------------------------------------------------------------------------------------------------
	\section*{References}
	\begin{frame}[noframenumbering]
		\frametitle{References}
		\printbibliography
		%\nocite{*}
	\end{frame}
	%
	%===================================================================================================================
	\appendix
	\section{\appendixname}
	\begin{frame}
		\frametitle{Appendix - Table of Contents}
		% a brief summary of additional content
		\begin{itemize}
			\item Algorithm, page \pageref{app:algorithm}
			\item Cement paste tests\cite{ts1ds}, water-cement ratio 0.40, D = 25 mm, page \pageref{app:cem25}
			\item Cement paste tests\cite{ts1ds}, water-cement ratio 0.40, D = 70 mm, page \pageref{app:cem70}
			\item Reference tests, air\cite{ts5ds}, D = 25 mm, page \pageref{app:air25}
			\item Reference tests, air\cite{ts5ds}, D = 50 mm, page \pageref{app:air50}
			\item Reference tests, air\cite{ts5ds}, D = 70 mm, page \pageref{app:air70}
			\item Reference tests, water\cite{ts6ds}, D = 25 mm, page \pageref{app:water25}
			\item Reference tests, water\cite{ts6ds}, D = 50 mm, page \pageref{app:water50}
			\item Reference tests, water\cite{ts6ds}, D = 70 mm, page \pageref{app:water70}
		\end{itemize}
	\end{frame}
	%
	%-------------------------------------------------------------------------------------------------------------------
	\begin{frame}
		\frametitle{Appendix - Algorithm part I}\label{app:algorithm}
		% pseudo-code, mathematical description of the method
		$\mathbf{x} \in \mathbb{R}^N$ \ldots signal amplitude array\\
		$F_s \in \mathbb{R}$ \ldots sampling frequency\\
		$z \in \mathbb{R}$ \ldots zero-padding factor, multiple of signal length $N$\\
		\vspace*{1em}
		$F_{lim}$ = ESTIMATE\_FREQUENCY\_BAND($\mathbf{x}$, $F_s$, $z$, $t_P$, $t_F$)\\
		\begin{algorithmic}[1]
			\Ensure $0.8 \leq t_P \leq 0.99, \; t_P \in \mathbb{R}$
				\Comment{signal power threshold factor}
			\Ensure $0.5 \leq t_F \leq 0.9, \; t_F \in \mathbb{R}$
				\Comment{frequency threshold factor}
			\State $\mathbf{F, S_{xx}} \gets PSD(\mathbf{x}, F_s, \lfloor N \, z \rfloor)$
				\Comment{unilateral power spectral density (PSD) and frequencies}
			\State $N_{bin} \gets |\mathbf{F}|$
				\Comment{number of frequency bins}
			\State $L_{\nu} \gets \lfloor N_{bin} \, t_F \rfloor$
				\Comment{length of noise measurement section}
			\State $L_s \gets N_{bin} - L_{\nu}$
				\Comment{length of signal section}
		\end{algorithmic}
		\vspace*{1em}
		\small Continued on next page.
	\end{frame}
	%
	%-------------------------------------------------------------------------------------------------------------------
	\begin{frame}
		\frametitle{Appendix - Algorithm part II}
		% pseudo-code, mathematical description of the method
		
		\begin{algorithmic}[1]
			\setcounter{ALG@line}{4}
			\State $P_x \gets \frac{1}{N} \, \sum\limits_{n=0}^{N} x_n^2$
				\Comment{estimate power of signal in noise}
			\State $P_{\nu} \gets \dfrac{N_{bin}}{L_{\nu}} \, \sum\limits_{k=L_s}^{N_{bin}-1} S_{xx,k}$
				\Comment{estimate noise power}
			\State $P_s \gets P_x - P_{\nu}$
				\Comment{estimate signal power}
			\State $C_k \gets \sum\limits_{m=0}^{k} S_{xx,k}, \; k = 0,\ldots,N_{bin}-1$
				\Comment{cumulative sum of the PSD}
			\State $U_m \gets k, \; \forall \, (C_k \neq C_j, \; k \neq j), \; k,j = 0,\ldots,N_{bin}-1$
				\Comment{bin indices for unique values of $\mathbf{C}$}
			\State $\mathbf{x_{int}} \gets C_k, \; k \in U$
				\Comment{unique sequence of interpolation supports}
			\State $\mathbf{y_{int}} \gets F_k, \; k \in U$
				\Comment{sequence of interpolation support amplitudes}
			\State $F_{lim} \gets INTERP1(\mathbf{x_{int}}, \mathbf{y_{int}}, P_s \, t_P)$
				\Comment{piece-wise, linear interpolation}
		\end{algorithmic}
		\vspace*{.5em}
		\small Continued on next page.
	\end{frame}
	%
	%-------------------------------------------------------------------------------------------------------------------
	\begin{frame}
		\frametitle{Appendix - Algorithm part III}
		% pseudo-code, mathematical description of the method
		\begin{algorithmic}[1]
			\setcounter{ALG@line}{12}
			\State $k_{lim} \gets \inf(arg(\mathbf{F} \geq F_{lim}))$
				\Comment{find bin index close to frequency limit}
			\State $L_s \gets \min(2 \, k_{lim}, 0.5 \, N_{bin})$
				\Comment{update length of signal section}
			\State $L_{\nu} \gets N_{bin} - L_s$
				\Comment{update length of noise measurement section}
			\State $P_{\nu} \gets \dfrac{N_{bin}}{L_{\nu}} \, \sum\limits_{k=L_s}^{N_{bin}-1} S_{xx,k}$
				\Comment{update estimate noise power}
			\State $P_s \gets P_x - P_{\nu}$
				\Comment{update estimate signal power}
			\State $F_{lim} \gets INTERP1(\mathbf{x_{int}}, \mathbf{y_{int}}, P_s \, t_P)$
				\Comment{piece-wise, linear interpolation}
			\State \Return $F_{lim}$
				\Comment{return estimate frequency band limit}
		\end{algorithmic}
		\vspace*{1em}
		\small Function $PSD$, see program code\cite{progcode}, function file \texttt{tool\_est\_dft.m} \\
		Function $INTERP1$, see GNU Octave function documentation \texttt{interp1} \\
		Function ESTIMATE\_FREQUENCY\_BAND, see program code, function file \texttt{tool\_est\_fqband.m}
	\end{frame}
	%
	%-------------------------------------------------------------------------------------------------------------------
	\begin{frame}
		\frametitle{Appendix - Cement paste tests, water-cement ratio 0.40, D = 25 mm}\label{app:cem25}
		% place graphs and images in the left column
		% comment on results in the right column
		\begin{columns}[t]
			\begin{RIPcolleft}
				\begin{figure}
					\includegraphics[height=55mm,trim= 0mm 0mm 0mm 20mm] {ts_DS_ts1_wc040_d25_4.png}
				\end{figure}
			\end{RIPcolleft}
			\begin{RIPcolright}
				\textbf{Data set:} \\
				\begin{itemize}
					\item ts1\_wc040\_d25\_4 \cite{ts1ds}
					\item 288 signals per channel
				\end{itemize}
				\textbf{Observations:} \\
				\begin{itemize}
					\item S-wave: "hill" around maturity of 5.5 hr
				\end{itemize}
			\end{RIPcolright}
		\end{columns}
	\end{frame}
	%
	%-------------------------------------------------------------------------------------------------------------------
	\begin{frame}
		\frametitle{Appendix - Cement paste tests, water-cement ratio 0.40, D = 25 mm}
		% place graphs and images in the left column
		% comment on results in the right column
		\begin{columns}[t]
			\begin{RIPcolleft}
				\begin{figure}
					\includegraphics[height=55mm,trim= 0mm 0mm 0mm 20mm] {nat_DS_ts1_wc040_d25_4_SID_72.png}
				\end{figure}
			\end{RIPcolleft}
			\begin{RIPcolright}
				\textbf{Data set:} \\
				\begin{itemize}
					\item ts1\_wc040\_d25\_4 \cite{ts1ds}
					\item signal \#72 of 288, maturity = 6 hr
				\end{itemize}
				\textbf{Observations:} \\
				\begin{itemize}
					\item S-wave, PSD: small side lobe at $F_{lim} \approx 180$ kHz
				\end{itemize}
			\end{RIPcolright}
		\end{columns}
	\end{frame}
	%
	%-------------------------------------------------------------------------------------------------------------------
	\begin{frame}
		\frametitle{Appendix - Cement paste tests, water-cement ratio 0.40, D = 70 mm}\label{app:cem70}
		% place graphs and images in the left column
		% comment on results in the right column
		\begin{columns}[t]
			\begin{RIPcolleft}
				\begin{figure}
					\includegraphics[height=55mm,trim= 0mm 0mm 0mm 20mm] {ts_DS_ts1_wc040_d70_4.png}
				\end{figure}
			\end{RIPcolleft}
			\begin{RIPcolright}
				\textbf{Data set:} \\
				\begin{itemize}
					\item ts1\_wc040\_d70\_4 \cite{ts1ds}
					\item 288 signals per channel
				\end{itemize}
				\textbf{Observations:} \\
				\begin{itemize}
					\item P-wave: maximum $F_{lim} \approx 110$ \% of $F_{nat}$
					\item S-wave: maximum $F_{lim} \approx 60$ \% of $F_{nat}$
					\item P-wave: "hill" between 10 and 18 hr
				\end{itemize}
			\end{RIPcolright}
		\end{columns}
	\end{frame}
	%
	%-------------------------------------------------------------------------------------------------------------------
	\begin{frame}
		\frametitle{Appendix - Cement paste tests, water-cement ratio 0.40, D = 70 mm}
		% place graphs and images in the left column
		% comment on results in the right column
		\begin{columns}[t]
			\begin{RIPcolleft}
				\begin{figure}
					\includegraphics[height=55mm,trim= 0mm 0mm 0mm 20mm] {nat_DS_ts1_wc040_d70_4_SID_144.png}
				\end{figure}
			\end{RIPcolleft}
			\begin{RIPcolright}
				\textbf{Data set:} \\
				\begin{itemize}
					\item ts1\_wc040\_d70\_4 \cite{ts1ds}
					\item signal \#144 of 288, maturity = 12 hr
				\end{itemize}
				\textbf{Observations:} \\
				\begin{itemize}
					\item P-wave, PSD: additional side lobe at $F_{lim} \approx 525$ kHz
				\end{itemize}
			\end{RIPcolright}
		\end{columns}
	\end{frame}
	%
	%-------------------------------------------------------------------------------------------------------------------
	\begin{frame}
		\frametitle{Appendix - Reference tests, air, D = 25 mm}\label{app:air25}
		% place graphs and images in the left column
		% comment on results in the right column
		\begin{columns}[t]
			\begin{RIPcolleft}
				\begin{figure}
					\includegraphics[height=55mm,trim= 0mm 0mm 0mm 20mm] {ts_DS_ts5_d25_b16_v800.png}
				\end{figure}
			\end{RIPcolleft}
			\begin{RIPcolright}
				\textbf{Data set:} \\
				\begin{itemize}
					\item ts5\_d25\_b16\_v800 \cite{ts5ds}
					\item 10 repetitions
				\end{itemize}
				\textbf{Observations:} \\
				\begin{itemize}
					\item P-wave: $F_{lim} \approx 60$ \% of $F_{nat}$
					\item S-wave: $F_{lim} \approx 40$ \% of $F_{nat}$
				\end{itemize}
			\end{RIPcolright}
		\end{columns}
	\end{frame}
	%
	%-------------------------------------------------------------------------------------------------------------------
	\begin{frame}
		\frametitle{Appendix - Reference tests, air, D = 25 mm}
		% place graphs and images in the left column
		% comment on results in the right column
		\begin{columns}[t]
			\begin{RIPcolleft}
				\begin{figure}
					\includegraphics[height=55mm,trim= 0mm 0mm 0mm 20mm] {nat_DS_ts5_d25_b16_v800_SID_5.png}
				\end{figure}
			\end{RIPcolleft}
			\begin{RIPcolright}
				\textbf{Data set:} \\
				\begin{itemize}
					\item ts5\_d25\_b16\_v800 \cite{ts5ds}
					\item Signal \#5 of 10 repetitions
				\end{itemize}
			\end{RIPcolright}
		\end{columns}
	\end{frame}
	%
	%-------------------------------------------------------------------------------------------------------------------
	\begin{frame}
		\frametitle{Appendix - Reference tests, air, D = 50 mm}\label{app:air50}
		% place graphs and images in the left column
		% comment on results in the right column
		\begin{columns}[t]
			\begin{RIPcolleft}
				\begin{figure}
					\includegraphics[height=55mm,trim= 0mm 0mm 0mm 20mm] {ts_DS_ts5_d50_b16_v800.png}
				\end{figure}
			\end{RIPcolleft}
			\begin{RIPcolright}
				\textbf{Data set:} \\
				\begin{itemize}
					\item ts5\_d50\_b16\_v800 \cite{ts5ds}
					\item 10 repetitions
				\end{itemize}
				\textbf{Observations:} \\
				\begin{itemize}
					\item P-wave: $F_{lim} \approx 60$ \% of $F_{nat}$
					\item S-wave: $F_{lim} \approx 50$ \% of $F_{nat}$
				\end{itemize}
			\end{RIPcolright}
		\end{columns}
	\end{frame}
	%
	%-------------------------------------------------------------------------------------------------------------------
	\begin{frame}
		\frametitle{Appendix - Reference tests, air, D = 50 mm}
		% place graphs and images in the left column
		% comment on results in the right column
		\begin{columns}[t]
			\begin{RIPcolleft}
				\begin{figure}
					\includegraphics[height=55mm,trim= 0mm 0mm 0mm 20mm] {nat_DS_ts5_d50_b16_v800_SID_5.png}
				\end{figure}
			\end{RIPcolleft}
			\begin{RIPcolright}
				\textbf{Data set:} \\
				\begin{itemize}
					\item ts5\_d50\_b16\_v800 \cite{ts5ds}
					\item Signal \#5 of 10 repetitions
				\end{itemize}
			\end{RIPcolright}
		\end{columns}
	\end{frame}
	%
	%-------------------------------------------------------------------------------------------------------------------
	\begin{frame}
		\frametitle{Appendix - Reference tests, air, D = 70 mm}\label{app:air70}
		% place graphs and images in the left column
		% comment on results in the right column
		\begin{columns}[t]
			\begin{RIPcolleft}
				\begin{figure}
					\includegraphics[height=55mm,trim= 0mm 0mm 0mm 20mm] {ts_DS_ts5_d25_b16_v800.png}
				\end{figure}
			\end{RIPcolleft}
			\begin{RIPcolright}
				\textbf{Data set:} \\
				\begin{itemize}
					\item ts5\_d70\_b16\_v800 \cite{ts5ds}
					\item 10 repetitions
				\end{itemize}
				\textbf{Observations:} \\
				\begin{itemize}
					\item P-wave: $F_{lim} \approx 65$ \% of $F_{nat}$
					\item S-wave: $F_{lim} \approx 40$ \% of $F_{nat}$
				\end{itemize}
			\end{RIPcolright}
		\end{columns}
	\end{frame}
	%
	%-------------------------------------------------------------------------------------------------------------------
	\begin{frame}
		\frametitle{Appendix - Reference tests, air, D = 70 mm}
		% place graphs and images in the left column
		% comment on results in the right column
		\begin{columns}[t]
			\begin{RIPcolleft}
				\begin{figure}
					\includegraphics[height=55mm,trim= 0mm 0mm 0mm 20mm] {nat_DS_ts5_d70_b16_v800_SID_5.png}
				\end{figure}
			\end{RIPcolleft}
			\begin{RIPcolright}
				\textbf{Data set:} \\
				\begin{itemize}
					\item ts5\_d70\_b16\_v800 \cite{ts5ds}
					\item Signal \#5 of 10 repetitions
				\end{itemize}
			\end{RIPcolright}
		\end{columns}
	\end{frame}
	%
	%-------------------------------------------------------------------------------------------------------------------
	\begin{frame}
		\frametitle{Appendix - Reference test, water, D = 25 mm}\label{app:water25}
		% place graphs and images in the left column
		% comment on results in the right column
		\begin{columns}[t]
			\begin{RIPcolleft}
				\begin{figure}
					\includegraphics[height=55mm,trim= 0mm 0mm 0mm 20mm] {ts_DS_ts6_d25_b16_v800.png}
				\end{figure}
			\end{RIPcolleft}
			\begin{RIPcolright}
				\textbf{Data set:} \\
				\begin{itemize}
					\item ts6\_d25\_b16\_v800 \cite{ts6ds}
					\item 10 repetitions
				\end{itemize}
				\textbf{Observations:} \\
				\begin{itemize}
					\item P-wave: $F_{lim} \approx 130$ \% of $F_{nat}$
					\item S-wave: $F_{lim} \approx 75$ \% of $F_{nat}$
				\end{itemize}
			\end{RIPcolright}
		\end{columns}
	\end{frame}
	%
	%-------------------------------------------------------------------------------------------------------------------
	\begin{frame}
		\frametitle{Appendix - Reference test, water, D = 25 mm}
		% place graphs and images in the left column
		% comment on results in the right column
		\begin{columns}[t]
			\begin{RIPcolleft}
				\begin{figure}
					\includegraphics[height=55mm,trim= 0mm 0mm 0mm 20mm] {nat_DS_ts6_d25_b16_v800_SID_5.png}
				\end{figure}
			\end{RIPcolleft}
			\begin{RIPcolright}
				\textbf{Data set:} \\
				\begin{itemize}
					\item ts6\_d25\_b16\_v800 \cite{ts6ds}
					\item Signal \#5 of 10 repetitions
				\end{itemize}
				\textbf{Observations:} \\
				\begin{itemize}
					\item P-wave, PSD: additional side lobe at $F \approx 1500$ kHz
				\end{itemize}
			\end{RIPcolright}
		\end{columns}
	\end{frame}
	%
	%-------------------------------------------------------------------------------------------------------------------
	\begin{frame}
		\frametitle{Appendix - Reference test, water, D = 50 mm}\label{app:water50}
		% place graphs and images in the left column
		% comment on results in the right column
		\begin{columns}[t]
			\begin{RIPcolleft}
				\begin{figure}
					\includegraphics[height=55mm,trim= 0mm 0mm 0mm 20mm] {ts_DS_ts6_d50_b16_v800.png}
				\end{figure}
			\end{RIPcolleft}
			\begin{RIPcolright}
				\textbf{Data set:} \\
				\begin{itemize}
					\item ts6\_d50\_b16\_v800 \cite{ts6ds}
					\item 10 repetitions
				\end{itemize}
				\textbf{Observations:} \\
				\begin{itemize}
					\item P-wave: $F_{lim} \approx 100$ \% of $F_{nat}$
					\item S-wave: $F_{lim} \approx 75$ \% of $F_{nat}$
				\end{itemize}
			\end{RIPcolright}
		\end{columns}
	\end{frame}
	%
	%-------------------------------------------------------------------------------------------------------------------
	\begin{frame}
		\frametitle{Appendix - Reference test, water, D = 50 mm}
		% place graphs and images in the left column
		% comment on results in the right column
		\begin{columns}[t]
			\begin{RIPcolleft}
				\begin{figure}
					\includegraphics[height=55mm,trim= 0mm 0mm 0mm 20mm] {nat_DS_ts6_d50_b16_v800_SID_5.png}
				\end{figure}
			\end{RIPcolleft}
			\begin{RIPcolright}
				\textbf{Data set:} \\
				\begin{itemize}
					\item ts6\_d50\_b16\_v800 \cite{ts6ds}
					\item Signal \#5 of 10 repetitions
				\end{itemize}
				\textbf{Observations:} \\
				\begin{itemize}
					\item P-wave, PSD: additional side lobes at $F \approx 500$ kHz and $F \approx 1500$ kHz
				\end{itemize}
			\end{RIPcolright}
		\end{columns}
	\end{frame}
	%
	%-------------------------------------------------------------------------------------------------------------------
	\begin{frame}
		\frametitle{Appendix - Reference test, water, D = 70 mm}\label{app:water70}
		% place graphs and images in the left column
		% comment on results in the right column
		\begin{columns}[t]
			\begin{RIPcolleft}
				\begin{figure}
					\includegraphics[height=55mm,trim= 0mm 0mm 0mm 20mm] {ts_DS_ts6_d70_b16_v800.png}
				\end{figure}
			\end{RIPcolleft}
			\begin{RIPcolright}
				\textbf{Data set:} \\
				\begin{itemize}
					\item ts6\_d70\_b16\_v800 \cite{ts6ds}
					\item 10 repetitions
				\end{itemize}
				\textbf{Observations:} \\
				\begin{itemize}
					\item P-wave: $F_{lim} \approx 130$ \% of $F_{nat}$
					\item S-wave: $F_{lim} \approx 75$ \% of $F_{nat}$
				\end{itemize}
			\end{RIPcolright}
		\end{columns}
	\end{frame}
	%
	%-------------------------------------------------------------------------------------------------------------------
	\begin{frame}
		\frametitle{Appendix - Reference test, water, D = 70 mm}
		% place graphs and images in the left column
		% comment on results in the right column
		\begin{columns}[t]
			\begin{RIPcolleft}
				\begin{figure}
					\includegraphics[height=55mm,trim= 0mm 0mm 0mm 20mm] {nat_DS_ts6_d70_b16_v800_SID_5.png}
				\end{figure}
			\end{RIPcolleft}
			\begin{RIPcolright}
				\textbf{Data set:} \\
				\begin{itemize}
					\item ts6\_d70\_b16\_v800 \cite{ts6ds}
					\item Signal \#5 of 10 repetitions
				\end{itemize}
				\textbf{Observations:} \\
				\begin{itemize}
					\item P-wave, PSD: additional side lobe at $F \approx 560$ kHz
				\end{itemize}
			\end{RIPcolright}
		\end{columns}
	\end{frame}
	%-------------------------------------------------------------------------------------------------------------------
	\section*{Author information}
	\begin{frame}[noframenumbering]
		\frametitle{Author information}
		\RIPauthorinfo{}
	\end{frame}
	%-------------------------------------------------------------------------------------------------------------------
	\section*{Document license}
	\begin{frame}[noframenumbering]
		\frametitle{Document license}
		\expandafter\RIPcopyrightinfo\expandafter{\PresCopyrightType}
	\end{frame}
\end{document}
